\pagebreak

\begin{center}
	{\Large \bf FW-HTF Theme 2: Shaping the Socioeconomic Impact of Robotics and Autonomy in Healthcare}
\end{center}

\vspace{1 em}

\paragraph*{\Large Project Summary} 
The increasing adoption of robotics in healthcare poses far-reaching questions about how healthcare personnel will adapt to the use of robotic technology, the role of robot autonomy in a sector traditionally resistant to taking the human out of the loop, and how cognitive assistance for robot operators can be engineered to help avoid disparate impacts on the workforce.  This interdisciplinary project aims to address these questions by bringing together experts in medical robotics, psychology, economics, and medicine to study current and experimental robotic technologies for healthcare applications.  Healthcare robots have been adopted at various stages of maturity, ranging from established technologies like surgical robots and telepresence robots for remote consulting, to experimental but promising applications in rehabilitation, nursing, and home care robots.  They have the promise to improve patient outcomes, deliver high-quality care to rural and remote areas, allow workers whose jobs require physical interaction to telecommute, and provide physical assistance to reduce workplace injuries. And yet, there is a gap in understanding the socioeconomic impact of these technologies on patient-caregiver relationships, the job market, training, and job satisfaction of healthcare personnel.  This project hypothesizes that increasing robot adoption runs the risk of disparate impact on certain classes of healthcare worker, including biases from gender, age, and socioeconomic status.  Although the relationship between technology and workforce development is highly complex, the project hypothesizes that a major factor in this relationship is the amount of cognitive assistance provided by the user interface.  Using a mix of economic, psychological, and engineering methodologies, this research seeks to clarify the links between cognitive assistance and socioeconomic impact, and to develop best practices for engineering robot interfaces and worker training programs.

% Tele-operated robotic systems extend a human worker’s physical capabilities to perform manufacturing and maintenance tasks in remote, inaccessible, dangerous and/or hazardous environments. 
%This project aims to (1) develop novel teleoperation interface and training methodology that facilitate healthcare workers to acquire the motor skills of controlling complex motion coordinations frequently performed in nursing and assisting tasks; and (2) investigate the impacts of the proposed technology and training paradigm on healthcare jobs given the socio-cultural norms and biases. The state-of-the-art tele-nursing robots have been endowed with the physical structures for performing arm-hand coordination, bimanual coordination and loco-manipulation while perceiving environment through multi-model sensors from various perspectives. However, these capabilities haven't been exploited due the difficulties of learning the motion and perception mapping imposed by teleoperation interface. Measurement metrics for human and robot performances are not sufficient to quantify the characteristics of teleoperation interfaces, and compare the synergistic human-robot performance through the interfaces across tasks and along the user's motor skills progression. Novel training paradigm to adapt to the technology advancement hasn't been investigated in nursing education. On the social-economical side, it is unclear if the novel tele-nursing technology will significantly shift the job market due to socio-cultural biases (e.g., gender, age, etc) and the correlated education barriers. The socio-cultural biases may interfere with the design of the device and also the desire and ability to learn and use (tele)operation interfaces. For instance, it is possible that during the design process differences between genders are not examined. It is also possible that women are less interested (or even less able) to use (tele)operation interfaces due to gender bias or stereotype threat. Likewise, patients may be less likely to trust a tele-operated robotic system if it controlled by a female compared to a male.  

%To address the above technological, educational and social-economical issues, we propose to synergize research efforts from robotics, nursing education and Social psychology to achieve the following research objectives. Our Science and technology objectives aim to (1) Develop novel metrics to characterize the teleoperation interface (the operation complexity, motion mapping intuitivity, efficiency and predictability, perception transparency and balance, etc., and compare various teleoperation interfaces developed for a mobile humanoid nursing robots in motion coordination tasks. Based on the understanding of human-robot adaption through the teleoperation interfaces, we will further develop technologies that can (2) Shift the boundary between direct teleoperation and autonomous control based on the physical and mental status of the operator, (3) synthesize and convert sensory information to improve cognitive situation-awareness, and improve cognitive and physical skills of the operator, (4) Infer human teleoperator's contextual intent based on the knowledge of manipulation tasks and human motions, in order to automate appropriate low-level robot actions.  Our education objective aims to (1) Integrate ``cloud wisdom'' of multiple intelligence agents (human experts and AIs) into the skill evaluation and collaborative decision-making in terms cognitive augmentation, (2) Utilize interactive perception to engage the learning user to actively explore the robot's motion and perception capabilities. Our social-economy objectives aim to (1) determine nursing tasks where a tele-operated robotic system will be of assistance and design robotic system, (2) investigate perceptions of tele-operated robotic systems in relation to socio-cultural norms and biases to assist in the design and implementation. 

\vspace{0.5 em}

\paragraph*{\Large Intellectual Merit}
The intellectual merit of this proposal is to 1) clarify the connection between robotic technology adoption and socioeconomic impact in a broad range of healthcare sectors, and 2) to understand how cognitive and physical assistance in healthcare robots may impact job satisfaction and mitigate possible disparate impacts.  Moreover, novel technologies for intelligent user interfaces based on machine learning and artificial intelligence will be developed and evaluated on robotic telesurgery, telenursing, and in-home care applications.  Human subjects studies will help better understand how gender bias, stereotype threat, and other socio-cultural norms may be at play in robotics, which will further influence technology development and worker training. 

%This project addresses how multi-modality teleoperation interface affects the adaption of human motor behavior adapt to the motion and perception capabilities of a robotic physical embodiment. It develops novel methodologies to evaluate the adaption level by the quality of low-level motor skills and high-level task plan. It uniquely integrates collaborative skill evaluation method to teleoperation skill acquisition process, and guides the learning process by maximizing the expected perception and motion information gain.  Moreover, this will be some of the first work to explicitly examine the effects that gender has on: a) perceptions of tele-operated robotic systems, individuals desire and ability to use tele-operated systems, and b) perceptions of the competence of the user of the tele-robotic system. It develops a unified framework that takes into consideration gender bias, stereotype threat, and other socio-cultural norms that may be at play, which will further influence the technology development and worker education. 

\vspace{0.5 em}

\paragraph*{\Large Broader Impacts}
This research will have the broader social impact of benefiting healthcare workers that operate or will operate robots in the course of their job duties, as well as other job sectors that are beginning to see workers interact with robots, like autonomous driving, manufacturing, agriculture, and household robots. The adoption of robots into these sectors can reduce the exposure of human personnel to dirty, dull, and dangerous tasks, and reduce the ergonomic and physical demands of certain tasks that have disparate impact on women and the elderly.  Research will be closely linked to graduate and undergraduate education for students from engineering, social sciences, and medical schools, and outreach to K-12 students and the general public will also be pursued. 

\pagebreak
