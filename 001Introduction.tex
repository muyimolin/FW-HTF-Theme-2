
%-------------------------------------------------------------------------
\section{Introduction and Significance}\label{sec:intro}
%-------------------------------------------------------------------------

Major challenges facing the healthcare system in the US and worldwide include an aging population, high costs of treatment, increasing clerical burdens, and a shortage of doctors and nurses. These problems call for forward-thinking socio-technological solutions, and one approach currently being considered is to adopt robotics into various aspects of healthcare.  Robots have the potential to dramatically change the practice of clinical care in the near future as hospitals begin to adopt surgical robots to enhance surgeon dexterity in minimally invasive procedures, telepresence robots used for remote consultation, food and medicine delivery robots, assistive robots for elder care, patient lifting robots, and teleoperated robots for clinical care in quarantine rooms.  However, the socioeconomic implications of robot adoption on the state of patient care and the healthcare workforce are still poorly understood.  From the point of view of patients and caregivers, the use of robotics to augment or replace job responsibilities may be considered to be hostile, degrading, and even discriminatory. At these early stages of technology development, the time is ripe to carefully consider these implications so that their positive and negative consequences can be better predicted and hopefully controlled to preserve broad access to productive and meaningful work.

The purpose of this project is to investigate two general thematic questions regarding the adoption of robotics in healthcare.  First, how can cognitive assistance in these robots, in the form of autonomy or partial autonomy, be designed to satisfy workers' occupational needs while maintaining adequate technical performance?   Second, how will the increasing adoption of robots  potentially benefit or harm healthcare workers and the broader healthcare system?   These themes are {\em multifaceted}, involving engineering, psychological, and socioeconomic issues, and they are {\em overlapping}, since the nature of cognitive assistance provided by the human interface will inevitably affect how workers adapt to robotic technology.  This three-year collaborative research project assembles an interdisciplinary team to address these questions using methodologies from economics, psychology, and engineering.  Although this project is focused on the context of healthcare, the results from this research will also be relevant to broader questions of embodied intelligent cognitive assistants at large.

To address these questions, the activities of this project are organized along three specific aims:

\begin{itemize}
\item \textbf{Aim 1: Develop frameworks for autonomous cognitive assistance in healthcare robots} --- Evaluate forms of cognitive assistance across a range of healthcare robots.  Study information displays for complex, multimodal sensor information and background knowledge.  Implement partially automated tasks that may be routine, difficult for direct control, or cooperative with humans.  Investigate algorithms for inferring human operator intent for coordination with humans, and adaptive task scheduling for cooperative human-robot teams.

\item \textbf{Aim 2: Model the socioeconomic impact of healthcare robots} --- Study the impact of robotic adoption on the economy and labor market.  Survey healthcare worker attitudes toward robotics and study the robot training process for telerobotic (non-autonomous) systems.  Characterize potential disparate impacts by location, gender, race, and age. 

\item \textbf{Aim 3: Model the interaction between autonomy and impact} --- Study human factors issues regarding the human-autonomy interface.  Characterize how autonomous features of healthcare robots may affect disparate impacts, and suggest best practices for future technology development.

%\item \textbf{Aim 1, Technology} ---  Study the Developed Motion-Perception Coordination: Compare developed teleoperated motion coordination across various teleoperation interfaces; Investigate the motion coordination strategy under passive, active, and interactive perception; Study the regularity and variability of the developed motion coordination across teleoperation interfaces, to reveal the adaption of human motor control to robot motion and perception capabilities. 

%\item \textbf{Aim 2, Education} --- Study the Motion Coordination Development Process: Develop performance evaluation metrics for human-robot teleoperation system; Study the progression of teleoperated motion coordination from novice to expert, and identify the milestones and thresholds of performance improvement; Investigate the evolution of motion primitives and task plan; Prescribe multi-modality cognitive augmentation to overcome the skill development thresholds; Prescribe physical-cognitive training tasks to develop transferable skills for perception-motion coordination. 

%\item \textbf{Aim 3, Social-economy} --- Compare the skill development processes among populations different in gender, race, and age; Evaluate the potential effects of cognitive function differences and socio-cultural biases on robot teleoperation skill acquisition. 

%\item \textbf{Aim 4} --- Integrate Aim 1-3; Develop novel teleoperation interfaces that can (1) support transparent and intuitive human-robot adaption, (2) convey to teleoperator most concise and relevant performance indices and task information to facilitate task development, (3) evaluate novel teleoperation skill training paradigm across different populations. 

\end{itemize}

The team has a unique capability to use experimental robotic testbeds for this research in the areas of {\em telesurgery}, {\em telenursing}, {\em rehabilitation}, and {\em in-home care}.  Although there are other forms of robotics being adopted in healthcare, these testbeds represent different points along the spectrum between existing (telesurgery, telemedicine) and promising future technologies (rehabilitation, telesurgery, in-home care).  Moreover, the endpoints of this spectrum roughly correspond to tele-robotics vs autonomous robots.  While existing surgical robots are directly operated by surgeons, in the near future, healthcare personnel will be operating robots at various levels of autonomy, ranging from direct teleoperation (no autonomy) to supervisory control (all low-level motor functions autonomous). Tele-robotic systems enable medical workers to perform surgical, nursing, assisting and rehabilitation tasks in remote, inaccessible, and/or quarantine environments. Through tele-robotic technologies, single operator can operate multiple robotic systems (fan-out), and rapid switching between robots dispatched to different service sites. On the other hand, autonomy provides cognitive assistance that aims to reduce the cognitive burden on a human operator, allowing them to focus on other tasks or to operate multiple robots. Increased automation leaves the human operator in the role of decision-maker rather than a pilot, which may make the job less tedious or more interesting. Automating the “dirty” parts of a job, or reducing the time spent in clerical work, could also lead to higher job satisfaction.  However, the use of autonomy faces resistance in the healthcare sector, which prefers to keep doctors and nurses in control of decision-making.

Although the relation between technology use and job satisfaction is complex, it can be reasonably hypothesized that technology is becoming a larger part of a worker’s job, their job satisfaction will be less negatively affected if the technology is more transparent or a worker is more comfortable with it. The ultimate study in this proposal (Aim 3) will be aimed at assessing whether healthcare robots that employ cognitive assistance for the operator will be viewed as more transparent and accepted by healthcare workers. These attitudes will be assessed with medical personnel and students across gender, age, and socioeconomic factors.

%From a novice to an expert teleoperator, human workers learn to adapt their motor control to the motion and perception capabilities of the robotic physical embodiments of their remote surrogates. 



% To reduce the workers' skill acquisition efforts, we propose to compare the motion coordination teleoperated through various interfaces, by users of developed and developing teleoperation skills. We investigate the regularity and variability of the developed motion coordinations across teleoperation interfaces, to reveal the underlying strategies of natural human motion and perception coordination. We also compare the developing motion coordinations during the training, to identify the milestones and threshold of the teleoperation skill progression.   


% in their , 

% our project will investigate the underlying strategies of natural human motion and perception coordination, and develop teleoperation interfaces that support intuitive perception and motion mapping.   


\paragraph*{Intellectual Merit.}
This research will investigate \textbf{novel user interfaces} that can fully synergize human and medical robots' physical and cognitive capabilities, \textbf{novel training paradigms} that can help medical personnel learn to use robots effectively, and \textbf{an enhanced understanding of socio-cultural norms and biases} in medical training and practice. 

\paragraph*{Broader Impacts.}  The outcomes from this research will help predict the potential impact of robotic technologies on healthcare jobs, and reduce bias based on location, gender, race, and age.

\paragraph*{Qualification of Investigators.}
The engineering team has extensive experience in robotics, and have developed robots for tele-surgery, rehabilitation, tele-nursing, and home care. The social science team consists of experts in gender bias, human factors, surgery, and nursing, including cross-disciplinary expertise in the implementation of autonomous interfaces and telemedicine. Robot platforms available for use in this research, built and operated by the co-PIs in prior work, include the Raven telesurgical robot, [WHAT REHABILITATION ROBOT], and two TRINA mobile manipulators. 

% Tele-robotic systems extend a medical worker's motion and perception capabilities to perform surgical, nursing, and rehabilitation tasks in remote, inaccessible, and/or hazardous environments. These tasks are risk-sensitive tasks usually involve intimate interaction with patients, and thus far can only be accomplished under direct teleoperation instead of autonomous control. Task performance heavily depends on the the human teleoperator's dexterous motion coordination skills, developed situational awareness, and domain knowledge for decision-making and situation evaluation. The state-of-the-arts medical robotic systems have been endowed with complex and capable motion and perception hardwares. The physical capabilities of medical robots and human teleoperators are bottlenecked by low-transparency teleoperation interfaces. To freely and efficiently control their remote surrogates, human workers need to devote significant efforts to learn the motion and perception mapping defined by the teleoperation interface. To address these needs, we propose to investigate motion/perception mapping , and develop a user interface to support intuitive robot control and multi-modality cognitive augmentation. 

% Our proposed project aims to (a) reduce the teleoperation control effort in dexterous and coordinated manipulation tasks, (b) facilitate novice workers to acquire the fine motor skills for operating complicate robotic systems to work on various manufacturing tasks, (c) propose novel worker training infrastructure and methodologies improve the skills and well-being of industrial workers, and (d) create safe, comfortable, and remotely accessible industrial job opportunities. We propose research objectives with the emphases on \textit{science and technology}, as well as \textit{education and social-economy}. Our Science and technology objectives aim to investigate theories and technologies that (1) Shift the boundary between direct teleoperation and autonomous control based on the physical and mental status of the operator, (2) synthesize and convert sensory information to improve cognitive situation-awareness, and improve cognitive and physical skills of the operator, (3) Infer human teleoperator's contextual intent based on the knowledge of manipulation tasks and human motions, in order to automate appropriate low-level robot actions. Our education and social-economy objective aims to (1) Integrate ``cloud wisdom'' in terms cognitive augmentation into the collaborative decision-making and motor learning process among multiple intelligence agents (human experts and AIs) through a democratic or weighted voting system, (2) Study the effect of cognitive-augmentation human-robot interface on industrial worker skill acquisition and mental/physical demands in training processes, (3) Develop novel training methodologies and infrastructure at home and in workplace, and (4) Investigate the social-economical impacts of such technology on small and large industries, and on societal job opportunity shift and regional and national job distribution.  


\paragraph*{Relevance to the goals of NSF's FW-HTF Big Idea}

\textcolor{red}{(a) transforming the frontiers of science and technology for human performance augmentation and workplace skill acquisition; (b) improving both worker quality of life and employer financial metrics; (c) enhancing the economic and social well-being of the country; and (d) addressing societal needs through research on learning and instruction in the context of augmentation.}


This will be the first work to explicitly examine the effects of age, gender, and socioeconomic status on: a) attitudes and acceptance of healthcare robot operation as part of job responsibilities, b) perceptions of one's own competence when operating a robot, and c) attitudes and acceptance of cognitive assistance provided by semi-autonomous interfaces.  The resulting theories of gender bias, stereotype threat, and other socio-cultural norms that may be at play, will inform the future of technology development and worker training. 
 
\paragraph*{Vision of success for the proposal}
\textcolor{red}{Specifically define the project goals and the definition of a successful outcome.}

Specific goals for the project include: 1) conduct an economic analysis of the impact of robotics on the healthcare labor market and healthcare industry, 2) survey the attitudes of nurses and doctors toward the use of robots as a conduit for or replacement for job responsibilities, 3) develop autonomous functions for tele-surgical and tele-nursing robots including at least one routine task and at least one difficult task, 4) conduct human subjects studies to understand worker attitudes toward shared control or semi-autonomous functions, 5) characterize how cognitive assistance in healthcare robots may affect disparate impact, and 6) suggest best practices for future robotic technology development.