\pagebreak
\setcounter{page}{1}
\setcounter{section}{0}

\section{Collaboration Plan}

\subsection{Summary}

Co-PIs from both engineering and the social sciences are co-located at each of the three partnering institutions (Duke University, Worcester Polytechnic Institute, and UCLA).  This will permit more frequent local activities to be conducted between larger-scale collaborative efforts across institutions.

The team at Duke (Hauser, Cummings, Shaw) will concentrate on the tele-nursing application and continues an existing collaboration between Duke’s Pratt School of Engineering and the School of Nursing. More broadly, the Duke team will contribute toward expertise in semi-autonomous decision making and human factors. The team will benefit from the relatively close proximity between the School of Engineering and the School of Nursing, and the TRINA robot has been transported twice already to the School of Nursing’s Simulation Lab to perform extended experiments.

The team at WPI (Li, Skorinko, Smith) will focus on the home care application and will contribute to expertise in human motor control and gender bias.  Both Duke and WPI have a shared TRINA mobile manipulator platform which will facilitate software and infrastructure sharing.

The team at UCLA (Rosen, ...) will focus on telesurgery and rehabilitation applications

\subsection{Meetings, Site Visits, and Student Exchange}

In-person meetings will be held between all of the Co-PIs and students at individual institutions at least monthly, and study personnel on individual Tasks will meet more frequently on an as-needed basis. 

Site visits will be held thrice yearly to help coordinate project goals and activities between institutions. These one-day workshops will involve discussion of past findings, current plans, and brainstorming. The budget for each institution includes travel for two personnel to attend these workshops, twice yearly. 

Activities in Aims 1 and 2 will be pursued largely by each institution in parallel, and using the medical domain in which the institution has greatest expertise. Coordination across institutions will be more intense in Aim 3 (Year 3), and we plan to host student exchanges during the summers or academic year in order to accomplish project goals.

\subsection{Specific roles of the Investigators}

PI Hauser (Duke) provides expertise in robotics and artificial intelligence, and he will coordinate project activities and lead the development of Task 1.1 (Automation of routine tasks) on the tele-nursing application.  He will also collaborate closely with Co-PI Cummings and Skorinko on designing and evaluating the human-autonomy interface.

Co-PI Cummings (Duke) provides expertise in human factors and human-robot interaction, and will led the development of Task 1.3 (Adaptive task scheduling) and 3.1 (Design of the human-autonomy interface).

Co-PI Li (WPI) provides expertise in human motor control and robotics, and will lead the development of Task 1.2 (Motion coordination assistance) on the in-home care and rehabilitation robotics applications.  She will also coordinate closely with Co-PI Skorinko on Task 2.3 during the robot evaluation portion of the study.

Co-PI Rosen (UCLA) provides expertise in robotic surgery and rehabilitation, and he will conduct Task 1.1 (Automation of routine tasks) in the context of the tele-surgery application, and will collaborate closely with Co-PI Li on Task 1.2 as applied to rehabilitation robotics.  

Co-PI Shaw (Duke) provides expertise in nursing and e-Health, and will lead the nursing component of Task 2.2 (Attitudes of personnel to robotics).

Co-PI Skorinko (WPI) provides expertise in psychology and gender bias, and will...

Co-PI Smith (WPI) provides expertise in economics, and will...
