\section*{Results from Prior NSF Support}

Dr. Hauser has been the PI of five NSF grants (RI-1218534, \$381,168, 8/2012–7/2015), (CAREER-1253553, \$481,151, 8/2013–7/2018), (SCH-1343940, \$686,411, 3/2014–2/2017), (IIS-1513221, \$73,025. 12/2014-11/2015) (NRI-1527826, \$472,712, 10/2015-9/2018). The CAREER and IIS RAPID grants, ``Cooperative Motion Planning for Human-Operated Robots'' and ``RAPID: Tele-Nursing Robots for Remote Treatment of Ebola Patients'' are the most closely related. \\
{\bf Intellectual Merit:} These grants studied methods for introducing greater robot intelligence in teleoperation systems, including the use of collision checking, motion planning, and grasp planning under the direction of human operators. Most results so far are on algorithms for satisfying the competing demands of both responsive and optimized motions. They have yielded 7 published publications so far (see references 14, 28, 29, 30, 40, 41, 42). \\
{\bf Broader Impacts:} A workshop on Algorithmic Human-Robot Interaction (AHRI) at the Human-Robot Interaction Conference and a AAAI Fall Symposium on Artificial Intelligence in Human-Robot Interaction (AI-HRI) were organized as part of the CAREER grant. One undergraduate REU student was funded for each summer of the CAREER grant, including one that resulted in a published paper (Eilering et al 2014), and a female student completed her PhD in Computer Science under support from this grant.  Five undergraduate independent studies were supervised as a part of the RAPID grant, which also contributed to enhancing institutional infrastructure for robotics research by providing hardware and software components for a teleoperated mobile manipulator.


Dr. Cummings served as PI of an NSF EAGER: Modeling Intent Communication Pathways for Human-Autonomous System Collaboration (\#1548417, \$299,990, 09/01/2015-08/31/2017).  The focus of this effort is to develop models of the signals, signs and symbols that inform skill, rule, and knowledge-based responses for both humans and autonomous systems attempting to cope with uncertainty. 
Summary of Results: This effort only just started so there are no empirical results as of yet. We are currently developing an experimental protocol to address intent communications in driverless car settings that include pedestrians. \textcolor{red}{TODO: update with new pubs}.  \\
{\bf Intellectual Merit:} This project determined how elements of the environment, the computation systems of robots, and unique traits of humans (including attention management) can be modeled to represent intent communication pathways that need to be instantiated in the system or the world around the system.\\
{\bf Broader Impact:} includes broader social impact in driverless car and manufacturing settings, and  there is significant potential to impact these rapidly developing domains.

Dr. Shaw has not received prior NSF support.
